\section{Abstract}
\textbf{Objective:} To compile a comprehensive summary of published human experience with levodopa given intravenously, with a focus on information required by regulatory agencies.

\textbf{Background:} While safe intravenous use of levodopa has been documented for over 50 years, an increase in regulatory supervision for pharmaceuticals given by a non-FDA-approved route has delayed or halted several studies. This is partially due to a US FDA requirement for comprehensive side effects data as part of an investigational new drug (IND) application, which is required if alternate routes of delivery of a drug are believed to carry a higher risk of adverse effects.

\textbf{Methods:} Over 200 articles were examined for details of administration, pharmacokinetics, and side effects of intravenous levodopa. Reports describing use of i.v. levodopa in humans were tabulated and summarized.

\textbf{Results:} We identified 139 original reports describing i.v. levodopa use in humans, with or without a peripheral decarboxylase inhibitor, beginning with psychiatric research in 1959-1960. 2651 subjects reportedly received i.v. levodopa, with outcome measures including parkinsonian signs, sleep variables, hormones, hemodynamics, CSF amino acid composition, regional cerebral blood flow, and behavior. Mean pharmacokinetic (PK) variables were summarized for 49 healthy subjects and 190 with Parkinson's disease. Side effects were those expected from clinical experience with oral levodopa. No articles reported deaths or induction of psychosis.

\textbf{Conclusion:} Over 2600 patients have received i.v. levodopa with a safety profile comparable to that seen with oral use.
\section{Methods}
The authors searched MEDLINE and OVID, reviewed selected books, searched toxicity databases, and followed references cited in those sources. Articles written completely in languages other than English, German, Italian, Spanish, or Portuguese were excluded. Search terms included (levodopa / L-dopa / DOPA) AND (intravenous / intravascular / infusion / injection / i.v.); limit to humans; search date through May, 2015. Information from oral and intraduodenal routes of L-DOPA administration was excluded except for PK/PD tables.  Studies in which IV levodopa was always coadministered with monoamine oxidase inhibitors (MAOIs) or catechol-O-methyltransferase (COMT) inhibitors were excluded.  Levodopa methyl ester \cite{3601092} and DL-dopa \cite{14430381} were included, but PK/PD calculations were corrected for the difference in molecular weights.  Co-administered drugs were reported if included by the authors.

We recorded total dose, maximum infusion rate, and pharmacokinetic (PK) and pharmacodynamic (PD) parameters where available, including steady state volume of distribution (VOD), clearance, distribution half life ($t_{\frac{1}{2}\alpha}$), and elimination half life ($t_{\frac{1}{2}}$ or $t_{\frac{1}{2}\beta}$), $E_{max}$, $EC_{50}$). Reported data were used to calculate any missing PK parameters where possible.  Additionally, any reports on efficacy were noted.  Side effect frequency was recorded if reported. The number of subjects and subject conditions (Parkinson disease versus healthy volunteers or other disease states) were recorded for each study. 

Average PK parameters were calculated across studies, weighted by the number of subjects.
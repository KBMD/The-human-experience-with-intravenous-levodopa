\section{Discussion}
The review of the existing literature strongly supports the safety of intravenous levodopa, which has been used in humans since 1959 (Pare and Sandler, 1959).  A comprehensive review of literature reports the human experience of intravenous levodopa in over 2600 human subjects.  With infusion rates as high as 5.0 mg/kg/hr and boluses as large as 200 mg, there are no recorded instances of fatal responses or of serious adverse events to intravenous levodopa, nor have there been documented cases of other serious side effects, such as psychosis, that might limit its use in humans.  

These dosing ranges correspond loosely with safety data in other species – according to the registry of toxic effects of chemical substances (RTECS), the lowest published toxic dose of levodopa in any non human species, by any route, is 2.5 mg/kg, causing a subtle behavioral effect on a learning measure in a mouse.  The lowest intravenous levodopa dose that is lethal to half the subjects that receive it (LD50) is “>100 mg/kg” in rats.  In mice, the LD50 ranges from 450 mg/kg (intravenously) to 4449 mg/kg (administered subcutaneously).  Typical human doses are in the range of only 1 mg/kg.  Thus, human studies with intravenous levodopa administer doses significantly lower than those dangerous to non human mammals, even though human subjects have not been reported to have significant toxicity when receiving significantly higher doses than the lowest toxic dose in rats.

Intravenous levodopa has similar efficacy and side effects to oral levodopa in humans.  These include GI (nausea, vomiting, and abdominal discomfort) and neuropsychiatric effects (dyskinesias).  Nausea and hypotension, the most common side effects of both IV and oral levodopa, are largely blocked by PDIs and are dependent on duration of previous treatment.  The other side effects mentioned were infrequent and neither serious nor life-threatening. The concerns that the intravenous form of levodopa might have more potent cardiovascular effects have been alleviated by co-administration with PDIs (Siddiqi et al., 2015).

The safety of IV levodopa is important for patients as well as for regulatory reasons, such as FDA approval.  Changing the route of administration of any drug in a study traditionally necessitates submitting an IND application if changing the route of administration “significantly increases the risks… associated with the use of the drug product” ( 21 CFR 312.2(b)(iii) ).  The data from our review of the literature suggest intravenous administration of levodopa does not significantly increase the associated risks of levodopa in comparison to oral administration.  Therefore, intravenous levodopa should continue to be used for research studies, as studies conducted throughout the past half-century have indicated the safety of IV levodopa administration in human patients.
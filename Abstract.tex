\section{Abstract}
\textbf{Objective:} To compile a comprehensive summary of published human experience with levodopa given intravenously, with a focus on information required by regulatory agencies.

\textbf{Background:} While safe intravenous use of levodopa has been documented for over 50 years, regulatory supervision for pharmaceuticals given by a route other than that approved by the U.S. Food and Drug Administration (FDA) has become increasingly cautious. If delivering a drug by an alternate route raises the risk of adverse events, an investigational new drug (IND) application is required, including a comprehensive review of toxicity data.

\textbf{Methods:} Over 200 articles referring to intravenous levodopa (IVLD) were examined for details of administration, pharmacokinetics, benefit and side effects.

\textbf{Results:} We identified 144 original reports describing IVLD use in humans, beginning with psychiatric research in 1959-1960 before the development of peripheral decarboxylase inhibitors. At least 2781 subjects have received IVLD, and reported outcomes include parkinsonian signs, sleep variables, hormones, hemodynamics, CSF amino acid composition, regional cerebral blood flow, cognition, perception and complex behavior. Mean pharmacokinetic variables were summarized for 49 healthy subjects and 190 with Parkinson's disease. Side effects were those expected from clinical experience with oral levodopa and dopamine agonists. No articles reported deaths or induction of psychosis.

\textbf{Conclusion:} At least 2781 patients have received i.v. levodopa with a safety profile comparable to that seen with oral administration.
\section{Results}
139 articles reporting intravenous levodopa administration were identified.  Most subjects with parkinsonism were diagnosed with idiopathic PD, while some studies included parkinsonism with a variety of etiologies varying from postencephalitic to vascular.  PD patients differed in their history of prior drug treatment before the studies with conditions including de novo, fluctuating, on-off, and stable.  Some subjects were treated with levodopa for conditions other than PD (see Table 1: Patient Populations and Response Parameters), including other movement disorders (dystonia, progressive supranuclear palsy), primary psychiatric disorders (schizophrenia, mood disorders, personality disorders),  endocrine disorders (diabetes mellitus, essential obesity, hypopituitarism), hepatic disease (alcoholic cirrhosis, steatohepatitis, hepatic encephalopathy), cardiac valvular disease, asthma, and neuroleptic malignant syndrome (NMS).  Healthy controls were also included in some studies.

Pharmacokinetic data were reported for a total of 251 human subjects (see Table 2: Pharmacokinetics of IV Levodopa).  Co-administration of a peripheral decarboxylase inhibitor (PDI) lowered the clearance and elimination half-life of intravenously administered levodopa, while there was no notable effect of PDIs on volume of distribution.

The pharmacodynamic data (see Table 3: Reports of Human Experience with IV Levodopa) obtained from the literature  surveyed a total of 2651 human subjects, with a significant variety of patient groups studied and a multitude of response parameters (see Table 1).  From these articles, no side effects were reported for a total of 1260 subjects.  The highest total dose was 4320 mg in one day, given to a patient with idiopathic PD and carcinoma of the retina.  The patient reported no side effects or adverse effects at this dose.  The highest single bolus dose was 200 mg, while the highest infusion rates were 5.0 mg/kg/hr.

Concomitantly administered peripheral decarboxylase inhibitors included carbidopa and benserazide.  Often, PDIs affected clearance and volume of distribution (as mentioned above), minimized gastrointestinal symptoms, and allowed subjects to be given lower doses of levodopa.    

Concomitant drugs were also listed, to help explain any side effects that might be caused by concomitant drug administration rather than by levodopa alone.  These included istradefylline (adenosine A2A receptor antagonist), amphetamines, methylphenidate, aminophylline, caffeine, terguride, estradiol, SKF38393 (a selective D-1 agonist), paroxetine, apomorphine, dextromethorphan, monoamine oxidase (MAO) inhibitors, and dantrolene.  

A variety of neurological, psychiatric, cardiovascular, and other physiological effects of levodopa  were monitored (see Table 1): electroencephalography, electrocardiography, rectal temperature, cerebral perfusion, metabolism, arterial blood pressure, autonomic function, psychosis, mood/anxiety ratings, human growth hormone levels, motor fluctuations, dyskinesias, motor UPDRS (Unified Parkinson’s Disease Rating Scale) scores, tapping rates, walking speed, local cerebral blood flow, heart rate, cardiovascular parameters, AVP (plasma arginine vasopressin), blood pressure, pulse, onset of REM and length of REM sleep stage, PR interval, PRA (plasma renin activity), arousal, concentration, digestion, liver and renal function, complete blood counts, and basic metabolic panels. \textit{We can omit this text if it's all in the table. /kjb}

There were no reported cases of death.  There were no instances of psychosis, even when attempting to elicit it in susceptible subjects (Goetz et al., 1998).  There were also no life-threatening events (serious adverse effects) following intravenous levodopa administration at high doses, regardless of whether a PDI was co-administered.  With co-administration of a PDI, the dosage range causing side effects (mainly nausea and asymptomatic hypotension) was a 0.5-2.0 mg/kg/hr infusion or a 45-150 mg bolus.  Without a co-administered PDI, side effects were reported at a 1.5-3.0 mg/kg/hr infusion or a 60-200 mg bolus. It should be noted that occurrence of side effects was more likely with higher doses, but other factors such as age, sex, disease severity, and prior treatment also played a role in side effects of levodopa.  

Other than these side effects found at high doses, milder or less frequent side effects included somnolence, lightheadedness, dizziness, headache, disorientation, euphoria, worsening or improvement of tics (in people with Tourette syndrome), weakness of the legs, increased dyskinesias and stiffness, REM sleep delay or shortening, improvement of asthma, increased blood pressure, tachycardia, higher cardiac output, benign premature ventricular contractions, urinary urgency, increased renal plasma flow, increase or decrease in plasma renin activity, increased sodium excretion, coldness of the limbs, syncope, “unpleasant sensation in head and abdomen,” reduction or increase in systolic blood pressure, heat sensation, vertigo, dystonia, chorea, emotional fluctuations and elevation in mood ratings, asthenia, insomnia, anxiety, increased "tensori", restlessness, disorientation and confusion, mild sedation, “slight and transient” orthostasis, “mild and transient” increase of PR interval, tachycardia, sweating, mild irritability, and stomachaches.  It is important to note that both side effects and efficacy depended strongly on subject factors including gender, age, past treatment, and disease state.  Also, dsykinesia was mentioned as a side effect only in patients with PD, and most often in those with a long history of previous levodopa treatment.

Motor benefits of levodopa in PD have been demonstrated conclusively. Additional reported benefits of intravenous levodopa treatment in PD included improved sleep (Hardie et al., 1984) and attenuation of early morning akinesia or dystonia (Juncos et al., 1987).  In other patient groups, benefits of intravenous levodopa included improvement of the comatose state in hepatic encephalopathy (Abramsky, Goldschmidt, 1974) and improvement in depression and somatoform symptoms (Ingvarsson et al., 1965).  Several studies found it to be superior to other treatment options for neuroleptic malignant syndrome (Nisijima et al., 1997).  More recently, intravenous levodopa treatment was found to alleviate the neuropsychiatric adverse effects (lethargy, hypersomnia, depression, agitation, akathisia, and confusion) associated with interferon-alpha (Sunami et al., 2002). All these effects are summarized in Table B.
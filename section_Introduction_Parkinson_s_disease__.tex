\section{Introduction}
	Parkinson’s disease (PD), a common cause of severe movement disorders and eventual neuropsychiatric disturbances, is associated with impairments in dopaminergic neurotransmission in the basal ganglia. Replacement of dopamine has been the cornerstone of treatment for PD; because dopamine itself does not cross the blood-brain barrier, its immediate precursor levodopa (L-3,4-dihydroxphenylalanine, L-DOPA) is used because it is not associated with this limitation (Birkmayer and Hornykiewicz, 1961; Hornykiewicz and Birkmayer, 1961; Hornykiewicz, 1963; Cotzias et al., 1967; Roe, 1997). Although purified levodopa was first known to be ingested by mouth in 1913 (Roe, 1997), it was initially used to treat human PD by intravenous rather than oral administration (Pare, Sandler, 1959; Birkmayer and Hornykiewicz, 1961).	 
Oral levodopa has become the preferred method of treatment clinically, but intravenous levodopa administration still holds advantages over the oral form for conducting research studies.  First, the rapid administration of intravenous levodopa is often necessary for certain study designs, including those focused on the pharmacokinetics and pharmacodynamics of the drug.  Additionally, intravenous administration leads to more predictable plasma levodopa concentration because oral medications have highly variable absorption characteristics, especially in PD patients (Bushmann et al., 1989), with differences in absorption based on variations in gender and age (Kompoliti 2002; Robertson, 1989).  Intravenous levodopa permits researchers to keep brain levodopa concentrations constant while assessing physiological responses over time.  Furthermore, clinical use of intravenous levodopa is sometimes necessitated by inability to tolerate oral medications, such as for PD patients during surgery or on total parenteral nutrition.  
Current U.S. FDA regulations have complicated the approval processes of using intravenous levodopa in research studies (Rascol et al., 2001).  Specifically, if risks of intravenous levodopa are significantly higher than those of oral levodopa, an IND (Investigational New Drug) application must be submitted.  Therefore, the overall goal of this paper is to facilitate research use of IV levodopa by compiling a literature review that comprehensively summarizes the human experience with intravenously administered levodopa.  We tabulate the extent of human exposure, side effects, benefits, and efficacy.  We also summarize pharmacokinetic (PK) and pharmacodynamic (PD) parameters from these studies. 

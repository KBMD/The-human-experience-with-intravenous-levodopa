\section{Discussion}
The existing literature strongly supports the safety of intravenous levodopa, which has been used in humans for more than half a century (Pare and Sandler, 1959).  Intravenous levodopa has been administered to over 2600 human subjects.  Despite infusion rates as high as 5.0 mg/kg/hr and boluses as large as 200 mg, there are no recorded instances of death or of other serious adverse events to intravenous levodopa, nor have there been documented cases of other serious side effects, such as psychosis, that might limit its use in humans.  

These conclusions are supported by safety data from other species. The registry of toxic effects of chemical substances (RTECS) reports the lowest published toxic dose of levodopa in any non human species, by any route, as 2.5 mg/kg, referring to a subtle behavioral effect on a learning measure in a mouse.  The lowest intravenous levodopa dose that was lethal to half the subjects that received it (LD50) was ``>100 mg/kg'' in rats.  In mice, the LD50 ranges from 450 mg/kg (intravenously) to 4449 mg/kg (administered subcutaneously).  Typical human doses are in the range of only 1 mg/kg.  Thus, human studies with intravenous levodopa administer doses significantly lower than those dangerous to nonhuman mammals. 

Intravenous levodopa has similar efficacy and side effects to oral levodopa.  These include GI (nausea, vomiting, and abdominal discomfort) and neuropsychiatric effects (sedation, dyskinesias).  Nausea and orthostatic hypotension, side effects of both IV and oral levodopa, are largely blocked by PDIs and are less common in patients accustomed to dopamimetic treatment.  The other side effects listed were infrequent and neither serious nor life-threatening. When given with adequate PDI pretreatment, intravenous levodopa has minimal if any cardiovascular effects (Siddiqi et al., 2015).

The safety of IV levodopa is important for patients as well as for regulatory review.  Changing the route of administration of any drug in a study traditionally necessitates submitting an IND application if changing the route of administration “significantly increases the risks… associated with the use of the drug product” ( 21 CFR \S 312.2(b)(iii) ).  The data from our review of the literature suggest intravenous administration of levodopa does not significantly increase the associated risks of levodopa in comparison to oral administration.  Therefore, intravenous levodopa should continue to be used for research studies, as studies conducted throughout the past half century support the safety of IV levodopa administration in human patients.
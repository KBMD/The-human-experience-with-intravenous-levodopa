\section{Conclusions}
Since 1959, at least 139 reports have described intravenous administration of levodopa to a total of 2651 subjects. The efficacy and adverse effect profile has been comparable to that of oral levodopa, especially when co-administered with peripheral decarboxylase inhibitors. Meanwhile, the pharmacokinetics have been more predictable, making this a useful tool for research studies. Parenteral administration is also beneficial for patients who are unable to safely take oral drugs, including patients undergoing long surgery, patients with hepatic encephalopathy, and patients with neuroleptic malignant syndrome. These results alleviate the concerns that intravenous levodopa should be considered to carry higher risks than the oral form for the purpose of monitoring by regulatory agencies.
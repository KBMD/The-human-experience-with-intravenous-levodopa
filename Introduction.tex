\section{Introduction}
Parkinson disease (PD), the second most common neurodegenerative disease, is associated with impairments in dopaminergic neurotransmission in the basal ganglia. Replacement of dopamine has been the cornerstone of treatment for PD. Because dopamine itself does not cross the blood-brain barrier (BBB), its immediate precursor levodopa (L-3,4-dihydroxphenylalanine, L-DOPA) is used because it does cross the BBB \cite{11763859,13954967,5334614}. Although purified levodopa was first ingested by mouth in 1913 \cite{Roe_1997}, it was first used for medical treatment by intravenous rather than oral administration \cite{14430381,11763859}.
	
Oral levodopa has become the preferred method of treatment clinically, but intravenous levodopa administration still holds advantages over the oral form for some research studies.  First, the rapid administration of intravenous levodopa is often necessary for certain study designs, including those focused on the pharmacokinetics and pharmacodynamics of the drug.  Additionally, intravenous administration leads to more predictable plasma levodopa concentration because oral medications have highly variable absorption characteristics, especially in PD patients \cite{2797454}, with differences in absorption based on variations in gender and age \cite{2775615,12011296}.  Intravenous levodopa permits researchers to keep brain levodopa concentrations constant while assessing physiological responses over time.  Furthermore, clinical use of intravenous levodopa has sometimes been used in patients who cannot tolerate oral medications, such as for PD patients during surgery or on total parenteral nutrition.  

Current U.S. FDA regulations have complicated the approval processes of using intravenous levodopa in research studies \cite{11176963}.  Specifically, an IND (Investigational New Drug) application must be submitted if risks of intravenous levodopa are significantly higher than those of oral levodopa (\href{http://www.accessdata.fda.gov/scripts/cdrh/cfdocs/cfcfr/CFRSearch.cfm?fr=312.2}{\S 21 CFR 312.2(b)(iii)}).  Therefore, the overall goal of this paper is to facilitate research use of IV levodopa by compiling a literature review that comprehensively summarizes the human experience with intravenously administered levodopa.  We tabulate the extent of human exposure, side effects, benefits, and efficacy.  We also summarize pharmacokinetic (PK) and pharmacodynamic (PD) parameters from these studies. 

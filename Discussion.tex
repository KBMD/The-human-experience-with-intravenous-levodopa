\section{Discussion}
The existing literature strongly supports the safety of intravenous levodopa, which has been used in humans for more than half a century \cite{14430381}.  Intravenous levodopa has been administered to over 2600 human subjects.  Despite infusion rates as high as 5.0 mg/kg/hr and boluses as large as 200 mg, there are no recorded instances of death or of other serious adverse effects of intravenous levodopa, nor have there been documented cases of other serious side effects, such as psychosis, that might limit its use in humans. Milder side effects, the most significant of which are nausea and vomiting, were most prominent with rapid infusions in the range of 1-2 mg/kg or 100-200 mg over less than 15 minutes \cite{5327616,Bbrung_1966,12865145,4555619,6540399,4880674}. 

These conclusions are supported by safety data from other species. The registry of toxic effects of chemical substances (RTECS) reports the lowest published toxic dose of levodopa in any non-human species as 2.5 mg/kg, referring to a subtle behavioral effect on a learning measure in a mouse, while higher doses have not been reported to cause similar effects in humans.  The lowest intravenous levodopa dose that was lethal to half of subjects (LD50) was ``>100 mg/kg'' in rats.  In mice, the LD50 ranges from 450 mg/kg (administered intravenously) to 4449 mg/kg (administered subcutaneously) \cite{rtecs}.  Typical human i.v. doses are in the range of only 1 mg/kg; thus, human studies with intravenous levodopa administer doses substantially lower than those dangerous to nonhuman mammals. 

Intravenous levodopa has similar efficacy and side effects as oral levodopa \cite{Connolly_2014} and dopamine agonists \cite{Bonuccelli_2008}.  These include GI (nausea, vomiting, and abdominal discomfort) and neuropsychiatric effects (sedation, dyskinesias).  Nausea and orthostatic hypotension, side effects of both IV and oral levodopa, are largely blocked by PDIs and are less common in patients accustomed to dopamimetic treatment.  The other side effects are infrequent and neither serious nor life-threatening \cite{Connolly_2014}. When given with adequate PDI pretreatment, intravenous levodopa has minimal if any cardiovascular effects \cite{Siddiqi2015}. The overall safety record is perhaps expected since L-DOPA is, after all, an amino acid present in the normal human brain. 

The safety of IV levodopa is important for patients but also for regulatory review.  Changing the route of administration of a drug in a research study necessitates submitting an IND application if changing the route of administration “significantly increases the risks \dots\ associated with the use of the drug product” (\href{http://www.accessdata.fda.gov/scripts/cdrh/cfdocs/cfcfr/CFRSearch.cfm?fr=312.2}{\S 21 CFR 312.2(b)(iii)}).  The data summarized here suggest that intravenous administration of levodopa does not significantly increase the risks of levodopa compared to oral administration.  In summary, studies conducted throughout the past half century support the safety of IV levodopa administration in human patients.
  